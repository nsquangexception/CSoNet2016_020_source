\section{Related work} \label{related-work}

ABSA approaches may be divided into three main categories: rule-based, supervised learning, and unsupervised learning.

% Rule-based
Rule-based approaches~\cite{bingliu,Ding} can perform quite well in a large number of domains.
They use a sentiment lexicon, expressions, rules of opinions, and the sentence parse tree to help classify the sentiment orientation on each aspect appeared in a review.
They also consider sentiment shifter words (i.e. \textit{not}, \textit{none}, \textit{nobody}, etc.).
However, these rule-based methods have a shortcoming in processing complex documents where the aspect is hidden in the sentence, and failing to group extracted aspect terms into categories.

% Supervised AE + SC
For the supervised learning approach, Wei and Gulla~\cite{Wei_Gulla} propose a hierarchical classification model to determine the dependency and the other relevant information in the sentence.
Jiang et al.~\cite{Jiang}, Boiy and Moens~\cite{Boiy} use dependency parser to generate a set of aspect-dependent features for classification, which weighs each feature based on the position of the feature relative to the target aspect in the parse tree.
Several other supervised learning models have been published, such as Hidden Markov Models~\cite{Jin_Wei}, SVMs~\cite{ABSA_SVM}, Conditional Random Fields~\cite{Choi_Yejin,Jakob_Iryna}.

% Unsupervised AE + SC
% Joint model
It has been demonstrated that a classifier trained from labeled data in one domain often performs poorly in another domain~\cite{BingLiubooks}.
Hence, unsupervised methods are often adopted to avoid this issue.
Several recent studies investigate statistical topic models which are unsupervised learning methods.
They assume that each document consists of a mixture of topics.
Specifically, Latent Dirichlet Allocation (LDA)~\cite{LDA_Blei}, Multi-Grain LDA model~\cite{Titov_Ryan} are used to model and extract topics from document collections.
A number of authors have considered the effects of topic models on ABSA task, such as the two-step approach~\cite{Brody_Elhadad}, joint sentiment/topic model~\cite{Lin_He}, and topic-sentiment mixture model~\cite{Mei}.
A recent study by Wang et al.~\cite{serbm} proposed the SERBM model which also jointly address these two tasks in an unsupervised setting.
% The two-step approach by Brody and Elhadad~\cite{Brody_Elhadad} to detect aspect-specific opinion words includes identifying aspects using topic models and identifying aspect-specific sentiment words by considering adjectives only.
% The joint sentiment/topic model was proposed by Lin and He~\cite{Lin_He} to separate aspect words and sentiment words.
% Mei et al.~\cite{Mei} proposed the topic-sentiment mixture model based on three separated models with the help of some external training data is proposed to extract aspect and sentiment words.

%Supervised learning is dependent on the training data and has difficulty to scale up to a large number of application domains.


